\documentclass[12pt]{article}
\usepackage[margin=0.8in]{geometry}
\usepackage{amsmath}
\usepackage{hyperref}

\title{Numerical analysis: Assignment 2}
\author{Niccolo Zuppichini}
\begin{document}

\maketitle
\section*{Exercise 1}

The code can be found in the folder \textit{code}. It has been written in Matlab. There's the main file which solves the given linear problem with size 9999 as specified on the assignment. \\

Moreover, I've checked my solution by assembling the matrix $A$, vector $b$ and solving it with an "in memory" Gauss Seidel implementation. Then I have adapted this code to be on place. Even though it is not required in the assignment, I have uploaded these codes as well. Running $main.m$ will display, in the console: \\
\begin{center}
	\textit{It took 5157 iterations to get the solution of size 9999 at machine error.}
\end{center}

\section*{Exercise 2}

To show what asked, I think it's better to split the matrix into two matrices easier to analyse. Note that the sum of \hyperref[sec:proof_sym]{symmetric} \hyperref[sec:proof_pd]{positive definite} matrices is symmetric positive definite too. \\
For n = 5 we have: \\
\\
$
\begin{bmatrix}
	6 & -2 & 0 & 0  & 1 \\
	-2 & 6 & -2 & 1 & 0 \\
	0 & -2 & 6 & -2 & 0\\
	0 & 1 & -2 & 6 & -2  \\
	 1 & 0 & 0 & -2 & 6 
\end{bmatrix}
 = 
 \begin{bmatrix}
	6 & -2 & 0 & 0  & 0 \\
	-2 & 6 & -2 & 0 & 0 \\
	0 & -2 & 6 & -2 & 0\\
	0 & 0 & -2 & 6 & -2  \\
	 0 & 0 & 0 & -2 & 6 
\end{bmatrix}
+ 
\begin{bmatrix}
	0 & 0 & 0 & 0  & 1 \\
	0 & 0 & 0 & 1 & 0 \\
	0 & 0 & 0 & 0 & 0\\
	0 & 1 & 0 & 0 & 0  \\
	 1 & 0 & 0 & 0 & 0 
\end{bmatrix}
= A + B 
$
\\
\\
The first matrix is a tridiagonal matrix and the second one an antidiagonal matrix with a zero-entry in the middle. This can be generalised for any odd number $n$. \\ 

By the \hyperref[sec:proof1]{spectral theorem}, a positive definite matrix has all the eigenvalues positive, moreover, if it is also symmetric then all the eigenvalues are real number and positive. \\


The first matrix $A$ is diagonally dominant and is clearly symmetric. The eigenvalues of the first matrix $A$ are given by the simple closed form (implemented as well in \textit{code/compute\_eig.m}): \\

$$\lambda_k = a - 2 \sqrt{bc} \; cos(\frac{k \pi}{n+1})$$

Where $a=6, b=c=-2$ then: \\

$$\lambda_k = 6 - 2 \sqrt{4} cos(\frac{k \pi}{n+1}) = 6 - 4 cos(\frac{k \pi}{n+1}) = $$

The cosine function returns values in the interval $[-1, 1]$ and therefore: \\

$$\lambda_k = 6 - 4 cos(\frac{k \pi}{n+1}) \geq 6 - 4 \geq 0 $$

Hence, all eigenvalues $\lambda_k$ are positive and therefore the first matrix (namely $A$) is positive definite. \\

The second matrix $B$ is symmetric too and its eigenvalues are, for any odd number of $n$ (0, 1, ..., 1)  hence are all real numbers and positive. It is, therefore, a symmetric positive definite matrix. 
\\

Both matrices $A$ and $B$ are symmetric positive definite, therefore their sum $A+B$ is symmetric positive definite as well. \\

\textbf{UPDATE:} I made an error on the eigenvalues of matrix $B$: the one I mentioned are the eigenvalues for identity matrix. But this is a flipped identity and its eigenvalues are (0, -1, -1,..., 1, 1) and therefore $B$ is not positive definite. \\

\subsection*{Proof: positive definite matrix has positive eigenvalues}
\label{sec:proof1}

By the spectral theorem we can decompose a matrix $A$ into $A = Q \Sigma Q^T$ where $Q$ is a matrix containing the eigenvectors of $A$ colomnwise and $\Sigma$ is a diagonal matrix containing the eigenvalues of $A$. Then, by definition, a positive definite matrix $A$ satisfies $x^T A x > 0 $. By the eigenvalue decomposition, $x^T A x = x^T Q \Sigma Q^T x > 0 \implies \sum_{i=1}^N \lambda_i (Q^T x)^2 > 0 $. Hence if all the eigenvalues $\lambda_i$ are positive then $A$ is positive definite. \\


\subsection*{Proof: the sum of positive definite matrix is positive definite}
\label{sec:proof_pd}
Given two positive definite matrices $A$ and $B$ then $A+B$ is positive definite too. \\ \\
\textbf{Proof:}
To prove this, it suffices to show that $x^T (A+B) x > 0 \forall x \in R^n $. \\

$$x^T (A+B) x = x^T A x + x^T B x > 0 $$ 


\subsection*{Proof: the sum of symmetric matrix is symmetric}
\label{sec:proof_sym}
A matrix is symmetric if $A=A^T$. Given two symmetric matrix $A$ and $B$ then $A+B$ is symmetric as well. \\

\textbf{Proof:} We want to show $(A+B)=(A+B)^T$. By the properties of matrix $(A+B)^T = A^T + B^T$ Since $A$ and $B$ are symmetric  $ A^T + B^T = A + B = (A+B) $. Therefore $(A+B)=(A+B)^T$.


\end{document}