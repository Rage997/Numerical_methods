\documentclass[12pt]{article}
\usepackage[margin=0.8in]{geometry}
\usepackage{amsmath}
\usepackage{hyperref}
\usepackage{graphicx}
\usepackage{float}
\usepackage{caption}
\usepackage{subcaption}

\title{Numerical analysis: Assignment 8}
\author{Niccolo Zuppichini}
\begin{document}

\maketitle
\section*{Exercise 1}

% Ref: https://mathworld.wolfram.com/Newton-CotesFormulas.html

The integral $\int_a^b f(x)$ can be approximated using a Lagrange polynomial of degree 2 as follows: \\

\begin{equation}
	\int_a^b f(x) \approx \int_{a=x_0}^{b=x3} P_2
	\label{eq:initial}
\end{equation}

The polynomial $P_2$ is given by:

% Not sure if the sum is up to 2
\begin{equation}
	P(x) = \sum_{j=0}^3 f(x_j) l_j(x)
	\label{eq:lagrange_poly}
	\caption{Lagrange polynomial of degree 2}
\end{equation}

with $l_j(x)$ defined as: 

\begin{equation}
	l_j(x) \prod_{j=0, m \neq j}^n \frac{x - x_m}{x_j - x_m}
	\label{eq:lagrange}
\end{equation}

Then by inserting Eq. \ref{eq:lagrange_poly} and Eq. \ref{eq:lagrange} into Eq. \ref{eq:initial}: 

\begin{equation}
	\int_a^b f(x) \approx \int_{a=x_0}^{b=x3} P = \int_{a=x_0}^{b=x_3} \sum_{j=0}^3 f(x_j) \prod_{j=0, m \neq j}^n \frac{x - x_m}{x_j - x_m}
	\label{eq:all_in}
\end{equation}

By rearranging Eq. \ref{eq:all_in} we get :
%(I haven't reported all the steps into latex but you can find them in the pdf \textit{exercise1\_calculus.pdf} \\

\begin{equation}
	\int_{a=x_1}^{b=x3} P = 
	\int_{x_0}^{x_3} f_1 \frac{x-x_2}{x_1 - x_2} \frac{x-x_3}{x_1 - x_3} +
	  f_2 \frac{x-x_1}{x_2 - x_1} \frac{x-x_3}{x_2 - x_3} +
	f_3 \frac{x-x_1}{x_3 - x_1} \frac{x-x_2}{x_3 - x_2}
	\label{eq:explicit_lagrange}
\end{equation}

By using the fact that the points are equidistant, Eq. \ref{eq:explicit_lagrange} becomes: 

\begin{equation}
	= \int_{x_1 - h}^{x_1+2h} 
	f_1 \frac{x^2 - x_2 x - x_3 x + x_2 x_3}{2h^2} +
	  f_2 \frac{x^2 - x_3 x - x_1 x + x_1 x_3}{-h^2} +
	f_3 \frac{x^2 - x_2 x - x_1 x + x_1 x_2}{2h^2} =
\end{equation}

\begin{equation}
	= \frac{1}{h^2} \int_{x_1 - h}^{x_1+2h} 
	f_1 \frac{x^2 - x_2 x - x_3 x + x_2 x_3}{2} -
	   f_2 (x^2 - x_3 x - x_1 x + x_1 x_3) +
	f_3 \frac{x^2 - x_2 x - x_1 x + x_1 x_2}{2} =
\end{equation}

By splitting the integrals: \\

\begin{equation}
	= \frac{1}{h^2} \bigg( 
	\frac{f_1}{2} \int_{x_1 - h}^{x_1+2h} x^2 - x_2 x - x_3 x + x_2 x_3 -
	   f_2 \int_{x_1 - h}^{x_1+2h} (x^2 - x_3 x - x_1 x + x_1 x_3) +
	\frac{f_3}{2} \int_{x_1 - h}^{x_1+2h} x^2 - x_2 x - x_1 x + x_1 x_2 \bigg) =
	\label{eq:last_step}
\end{equation}

Solving the three integrals gives the following relation :

\begin{equation}
	\int_a^b f(x) \approx \frac{4}{3} h (2 f_1 - f_2 + 2 f_3) +  \mathcal{O}(h^5)
\end{equation} 

The code has been implemented as required (i.e. dividing the interval (a,b) into 100 subpieces) in \textit{exercise1.py} and outputs: \\

\begin{center}
\begin{lstlisting}
Numerical solution:  0.9460279856300005

Analytical solution:  0.946083070367183

Absolute error:  5.50847371825203e-5	
\end{lstlisting}	
\end{center}

The console output of my code also verifies the order $\mathcal{O}(n^5)$ of accuracy. 


\section*{Exercise 2}

The implementation can be found in \textit{code/exercise2.py}. The output (for $j = 5)$ is : \\

\begin{center}
\begin{lstlisting}
Iterations:  5

Numerical solution:  4.10879206104986

Analytical solution:  4.670774270471606

Absolute error:  0.5619822094217461
\end{lstlisting}
	
\end{center}


\section*{Bonus exercise}

By the theorem of uniqueness of the interpolating polynomial: \\

\begin{equation}
	\begin{split}
			c(x) = q_1(x) + h_1(x) \\
			c(x) = q_2(x) + h_2(x) 
	\end{split}
\end{equation}

with: \\
\begin{equation}
	\begin{split}
		h_1(x) = \lambda (x-x_0) (x-x_1) (x-x_2) \\
		\lambda = \frac{f_3 - q_1(x_3)}{6h^3} \\
		h_2(x) = \mu (x-x_1) (x-x_2) (x-x_3) \\
		\mu = \frac{f_0 - q_2(x_0)}{-6h^3}		
	\end{split}
	\label{eq:relations}
\end{equation}

We need to show that: 
\begin{equation}
	\int_a^{a+3h} c(x) = \frac{1}{2} \int_a^{a+3h} q_1(x) + \frac{1}{2} \int_a^{a+3h} q_2(x)
\end{equation}

By inserting Eq. \ref{eq:relations} it follows: 

\begin{equation}
	\int_a^{a+3h} 2 c(x) = \int_a^{a+3h} q_1(x) + \int_a^{a+3h} q_2(x)
	\implies \int_a^{a+3h} h_1(x) + h_2(x) dx = 0
	\label{eq:new_relation}
\end{equation}

Therefore it suffices to show Eq. \ref{eq:new_relation} is equal to zero. However, I got lost into calculus. 

\end{document}